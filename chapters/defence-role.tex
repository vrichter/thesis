\begin{frame}{Recognizing Conversational Roles}
  \centering
    \begin{columns}[T] % align columns
      \begin{column}{.5\textwidth}
        \includegraphics[trim={4.1cm 0 0 0},clip,width=\textwidth]{conversational_group}
      \end{column}
      \begin{column}{.5\textwidth}
        \vspace{10pt}
        \textbf{RQ4: Conversational Roles}\\\vspace{5pt} 
        \onslide<2->{In \textcolor{myblue}{unconstrained interactions} in a smart environment: } \onslide<3->{How can the \textcolor{myblue}{conversational role} of an \textcolor{myblue}{agent} be recognized?}\\
      \vspace{10pt}
       \onslide<4->\(\rightarrow\) \hspace{5pt} Simple Models\\
       \vspace{5pt}
       \onslide<5->\(\rightarrow\) \hspace{5pt} Low-Level Features\\
       \vspace{5pt}
       \onslide<6->\(\rightarrow\) \hspace{5pt} Time Sequences\\
      \end{column}
    \end{columns}
  \pnote{14-5}
\end{frame}
\begin{frame}{Demonstration Corpus}
  \begin{columns}[T] % align columns
    \begin{column}{.5\textwidth}
      \centering
      \onslide<1->{\resizebox{\textwidth}{!}{%
        \def\svgwidth{1.5\textwidth}
        \input{generated/csra_map_annotations.pdf_tex}
      }}
    \end{column}%
    \begin{column}{.5\textwidth}
      \resizebox{\textwidth}{!}{%
        \scriptsize
        \begin{tikzpicture}
        \action<1->{\node (c) at (0,0)
        {
          \resizebox{\textwidth}{!}{%
          \def\svgwidth{\textwidth}
          \input{generated/defence_role_annotations_real-image.pdf_tex}
          }
        };}
        \action<1->{\node (c) at (0,0)
        {
          \resizebox{\textwidth}{!}{%
          \def\svgwidth{\textwidth}
          \input{generated/defence_role_annotations_real-groups.pdf_tex}
          }
        };}
        \action<2->{\node (c) at (0,0)
        {
          \resizebox{\textwidth}{!}{%
          \def\svgwidth{\textwidth}
          \input{generated/defence_role_annotations_real-speaker.pdf_tex}
          }
        };}
         \action<3->{\node (c) at (0,0)
        {
          \resizebox{\textwidth}{!}{%
          \def\svgwidth{\textwidth}
          \input{generated/defence_role_annotations_real-addressee.pdf_tex}
          }
        };}
         \action<4->{\node (c) at (0,0)
        {
          \resizebox{\textwidth}{!}{%
          \def\svgwidth{\textwidth}
          \input{generated/defence_role_annotations_real-sideparticipant.pdf_tex}
          }
        };}
         \end{tikzpicture}
      }
    \end{column}%
  \end{columns}
        57 Minutes of 2 Agents at \SI{15}{\Hz} \(\rightarrow\) \SI{102}{\kilo\nothing} Observations \\
        \footnotesize
        \onslide<2-> \textcolor{myred}{Speaker} (3408) 
        \onslide<3-> \textcolor{mygreen}{Addressee} (5689)
        \onslide<4-> \textcolor{myblue}{Side-Participant} (18358)
        \onslide<5-> \textcolor{black}{Non-Participant} (74213)
  \pnote{14-5}
\end{frame}
\begin{frame}{Simple Models \& High-Level Features}
  \centering
  \resizebox{.8\textwidth}{!}{%
    \begin{tikzpicture}
    \action<1->{\node (c) at (0,0)
    {
      \resizebox{.7\textwidth}{!}{%
      \def\svgwidth{0.98\textwidth}
         {\footnotesize
         \input{generated/rule-model-defence.pdf_tex}
         }
      }
    };}
    \action<2->{\node (c) at (0,-90pt)
    {
      \resizebox{.6\textwidth}{!}{%
      \def\svgwidth{0.98\textwidth}
         {\footnotesize
         \input{generated/bn-role-defence.pdf_tex}
         }
      }
    };}
     \end{tikzpicture}
  }
  \pnote{14-5}
\end{frame}
\begin{frame}{Low-Level Features \& Time Sequences}
      \centering
      \vspace{20pt}
      \resizebox{.8\textwidth}{!}{%
          \scriptsize
          \begin{tikzpicture}
          \node (b) at (0,0)
          {
            \resizebox{1.35\textwidth}{!}{%
            \def\svgwidth{.3\textwidth}
            \input{generated/nnl-defence-dense.pdf_tex}
            }
          };
          \action<1->{\node (c) at (30,0)
          {
            \resizebox{1.35\textwidth}{!}{%
            \def\svgwidth{.3\textwidth}
            \input{generated/nnl-defence-t2.pdf_tex}
            }
          };}
          \action<1->{\node (c) at (30,0)
          {
            \resizebox{1.35\textwidth}{!}{%
            \def\svgwidth{.3\textwidth}
            \input{generated/nnl-defence-t1.pdf_tex}
            }
          };}
          \action<1->{\node (c) at (30,0)
          {
            \resizebox{1.35\textwidth}{!}{%
            \def\svgwidth{.3\textwidth}
            \input{generated/nnl-defence-lstm.pdf_tex}
            }
          };}
           \end{tikzpicture}
        }
        \begin{itemize}[label=-]
          \item<2-> Train with different feature sets, number of layers, units in each layer
          \item<3-> Feature sets: High-level (4D as Bayes Model) \(\rightarrow\) Low-Level (148D)
          \item<4-> For lstm: sequences of 15 observations \(\approx\) \SI{1}{\second}
        \end{itemize}
  \pnote{14-5}
\end{frame}
\begin{frame}{Results - Confusion Matrices}
  \begin{columns}[T] % align columns
    \begin{column}{.65\textwidth}
      \resizebox{1.\textwidth}{!}{%
        \scriptsize
        \begin{tikzpicture}
          \action<3->{\node (b) at (0,0)
          {
            \resizebox{\textwidth}{!}{%
              % Created by tikzDevice version 0.12
% !TEX encoding = UTF-8 Unicode
\begin{tikzpicture}[x=1pt,y=1pt]
\definecolor{fillColor}{RGB}{255,255,255}
\path[use as bounding box,fill=fillColor,fill opacity=0.00] (0,0) rectangle (398.34,246.17);
\begin{scope}
\path[clip] (  0.00,  0.00) rectangle (398.34,246.17);
\definecolor{drawColor}{RGB}{255,255,255}
\definecolor{fillColor}{gray}{0.98}

\path[draw=drawColor,line width= 0.6pt,line join=round,line cap=round,fill=fillColor] (  0.00,  0.00) rectangle (398.34,246.17);
\end{scope}
\begin{scope}
\path[clip] ( 92.74,150.34) rectangle (240.04,224.42);
\definecolor{drawColor}{RGB}{255,255,255}

\path[draw=drawColor,line width= 0.6pt,line join=round] ( 92.74,160.93) --
	(240.04,160.93);

\path[draw=drawColor,line width= 0.6pt,line join=round] ( 92.74,178.56) --
	(240.04,178.56);

\path[draw=drawColor,line width= 0.6pt,line join=round] ( 92.74,196.20) --
	(240.04,196.20);

\path[draw=drawColor,line width= 0.6pt,line join=round] ( 92.74,213.84) --
	(240.04,213.84);

\path[draw=drawColor,line width= 0.6pt,line join=round] (113.78,150.34) --
	(113.78,224.42);

\path[draw=drawColor,line width= 0.6pt,line join=round] (148.85,150.34) --
	(148.85,224.42);

\path[draw=drawColor,line width= 0.6pt,line join=round] (183.92,150.34) --
	(183.92,224.42);

\path[draw=drawColor,line width= 0.6pt,line join=round] (218.99,150.34) --
	(218.99,224.42);
\definecolor{fillColor}{RGB}{77,175,74}

\path[fill=fillColor,fill opacity=0.39] ( 96.24,205.02) rectangle (131.32,222.66);
\definecolor{fillColor}{RGB}{228,26,28}

\path[fill=fillColor,fill opacity=0.47] (131.32,205.02) rectangle (166.39,222.66);
\definecolor{fillColor}{RGB}{228,26,28}

\path[fill=fillColor,fill opacity=0.31] (166.39,205.02) rectangle (201.46,222.66);
\definecolor{fillColor}{RGB}{228,26,28}

\path[fill=fillColor,fill opacity=0.16] (201.46,205.02) rectangle (236.53,222.66);
\definecolor{fillColor}{RGB}{228,26,28}

\path[fill=fillColor,fill opacity=0.13] ( 96.24,187.38) rectangle (131.32,205.02);
\definecolor{fillColor}{RGB}{77,175,74}

\path[fill=fillColor,fill opacity=0.75] (131.32,187.38) rectangle (166.39,205.02);
\definecolor{fillColor}{RGB}{228,26,28}

\path[fill=fillColor,fill opacity=0.27] (166.39,187.38) rectangle (201.46,205.02);
\definecolor{fillColor}{RGB}{228,26,28}

\path[fill=fillColor,fill opacity=0.17] (201.46,187.38) rectangle (236.53,205.02);
\definecolor{fillColor}{RGB}{228,26,28}

\path[fill=fillColor,fill opacity=0.11] ( 96.24,169.74) rectangle (131.32,187.38);
\definecolor{fillColor}{RGB}{228,26,28}

\path[fill=fillColor,fill opacity=0.45] (131.32,169.74) rectangle (166.39,187.38);
\definecolor{fillColor}{RGB}{77,175,74}

\path[fill=fillColor,fill opacity=0.51] (166.39,169.74) rectangle (201.46,187.38);
\definecolor{fillColor}{RGB}{228,26,28}

\path[fill=fillColor,fill opacity=0.25] (201.46,169.74) rectangle (236.53,187.38);
\definecolor{fillColor}{RGB}{228,26,28}

\path[fill=fillColor,fill opacity=0.10] ( 96.24,152.11) rectangle (131.32,169.74);
\definecolor{fillColor}{RGB}{228,26,28}

\path[fill=fillColor,fill opacity=0.11] (131.32,152.11) rectangle (166.39,169.74);
\definecolor{fillColor}{RGB}{228,26,28}

\path[fill=fillColor,fill opacity=0.16] (166.39,152.11) rectangle (201.46,169.74);
\definecolor{fillColor}{RGB}{77,175,74}

\path[fill=fillColor,fill opacity=0.96] (201.46,152.11) rectangle (236.53,169.74);
\definecolor{drawColor}{RGB}{0,0,0}

\node[text=drawColor,anchor=base,inner sep=0pt, outer sep=0pt, scale=  1.14] at (113.78,209.92) {0.32};

\node[text=drawColor,anchor=base,inner sep=0pt, outer sep=0pt, scale=  1.14] at (148.85,209.92) {0.40};

\node[text=drawColor,anchor=base,inner sep=0pt, outer sep=0pt, scale=  1.14] at (183.92,209.92) {0.23};

\node[text=drawColor,anchor=base,inner sep=0pt, outer sep=0pt, scale=  1.14] at (218.99,209.92) {0.06};

\node[text=drawColor,anchor=base,inner sep=0pt, outer sep=0pt, scale=  1.14] at (113.78,192.28) {0.03};

\node[text=drawColor,anchor=base,inner sep=0pt, outer sep=0pt, scale=  1.14] at (148.85,192.28) {0.70};

\node[text=drawColor,anchor=base,inner sep=0pt, outer sep=0pt, scale=  1.14] at (183.92,192.28) {0.19};

\node[text=drawColor,anchor=base,inner sep=0pt, outer sep=0pt, scale=  1.14] at (218.99,192.28) {0.08};

\node[text=drawColor,anchor=base,inner sep=0pt, outer sep=0pt, scale=  1.14] at (113.78,174.64) {0.01};

\node[text=drawColor,anchor=base,inner sep=0pt, outer sep=0pt, scale=  1.14] at (148.85,174.64) {0.38};

\node[text=drawColor,anchor=base,inner sep=0pt, outer sep=0pt, scale=  1.14] at (183.92,174.64) {0.45};

\node[text=drawColor,anchor=base,inner sep=0pt, outer sep=0pt, scale=  1.14] at (218.99,174.64) {0.17};

\node[text=drawColor,anchor=base,inner sep=0pt, outer sep=0pt, scale=  1.14] at (113.78,157.01) {0.00};

\node[text=drawColor,anchor=base,inner sep=0pt, outer sep=0pt, scale=  1.14] at (148.85,157.01) {0.01};

\node[text=drawColor,anchor=base,inner sep=0pt, outer sep=0pt, scale=  1.14] at (183.92,157.01) {0.06};

\node[text=drawColor,anchor=base,inner sep=0pt, outer sep=0pt, scale=  1.14] at (218.99,157.01) {0.93};
\end{scope}
\begin{scope}
\path[clip] ( 92.74, 54.51) rectangle (240.04,128.59);
\definecolor{drawColor}{RGB}{255,255,255}

\path[draw=drawColor,line width= 0.6pt,line join=round] ( 92.74, 65.09) --
	(240.04, 65.09);

\path[draw=drawColor,line width= 0.6pt,line join=round] ( 92.74, 82.73) --
	(240.04, 82.73);

\path[draw=drawColor,line width= 0.6pt,line join=round] ( 92.74,100.37) --
	(240.04,100.37);

\path[draw=drawColor,line width= 0.6pt,line join=round] ( 92.74,118.01) --
	(240.04,118.01);

\path[draw=drawColor,line width= 0.6pt,line join=round] (113.78, 54.51) --
	(113.78,128.59);

\path[draw=drawColor,line width= 0.6pt,line join=round] (148.85, 54.51) --
	(148.85,128.59);

\path[draw=drawColor,line width= 0.6pt,line join=round] (183.92, 54.51) --
	(183.92,128.59);

\path[draw=drawColor,line width= 0.6pt,line join=round] (218.99, 54.51) --
	(218.99,128.59);
\definecolor{fillColor}{RGB}{77,175,74}

\path[fill=fillColor,fill opacity=0.39] ( 96.24,109.19) rectangle (131.32,126.83);
\definecolor{fillColor}{RGB}{228,26,28}

\path[fill=fillColor,fill opacity=0.20] (131.32,109.19) rectangle (166.39,126.83);
\definecolor{fillColor}{RGB}{228,26,28}

\path[fill=fillColor,fill opacity=0.55] (166.39,109.19) rectangle (201.46,126.83);
\definecolor{fillColor}{RGB}{228,26,28}

\path[fill=fillColor,fill opacity=0.18] (201.46,109.19) rectangle (236.53,126.83);
\definecolor{fillColor}{RGB}{228,26,28}

\path[fill=fillColor,fill opacity=0.14] ( 96.24, 91.55) rectangle (131.32,109.19);
\definecolor{fillColor}{RGB}{77,175,74}

\path[fill=fillColor,fill opacity=0.25] (131.32, 91.55) rectangle (166.39,109.19);
\definecolor{fillColor}{RGB}{228,26,28}

\path[fill=fillColor,fill opacity=0.82] (166.39, 91.55) rectangle (201.46,109.19);
\definecolor{fillColor}{RGB}{228,26,28}

\path[fill=fillColor,fill opacity=0.12] (201.46, 91.55) rectangle (236.53,109.19);
\definecolor{fillColor}{RGB}{228,26,28}

\path[fill=fillColor,fill opacity=0.12] ( 96.24, 73.91) rectangle (131.32, 91.55);
\definecolor{fillColor}{RGB}{228,26,28}

\path[fill=fillColor,fill opacity=0.16] (131.32, 73.91) rectangle (166.39, 91.55);
\definecolor{fillColor}{RGB}{77,175,74}

\path[fill=fillColor,fill opacity=0.76] (166.39, 73.91) rectangle (201.46, 91.55);
\definecolor{fillColor}{RGB}{228,26,28}

\path[fill=fillColor,fill opacity=0.29] (201.46, 73.91) rectangle (236.53, 91.55);
\definecolor{fillColor}{RGB}{228,26,28}

\path[fill=fillColor,fill opacity=0.10] ( 96.24, 56.28) rectangle (131.32, 73.91);

\path[fill=fillColor,fill opacity=0.10] (131.32, 56.28) rectangle (166.39, 73.91);
\definecolor{fillColor}{RGB}{228,26,28}

\path[fill=fillColor,fill opacity=0.13] (166.39, 56.28) rectangle (201.46, 73.91);
\definecolor{fillColor}{RGB}{77,175,74}

\path[fill=fillColor] (201.46, 56.28) rectangle (236.53, 73.91);
\definecolor{drawColor}{RGB}{0,0,0}

\node[text=drawColor,anchor=base,inner sep=0pt, outer sep=0pt, scale=  1.14] at (113.78,114.09) {0.32};

\node[text=drawColor,anchor=base,inner sep=0pt, outer sep=0pt, scale=  1.14] at (148.85,114.09) {0.11};

\node[text=drawColor,anchor=base,inner sep=0pt, outer sep=0pt, scale=  1.14] at (183.92,114.09) {0.48};

\node[text=drawColor,anchor=base,inner sep=0pt, outer sep=0pt, scale=  1.14] at (218.99,114.09) {0.09};

\node[text=drawColor,anchor=base,inner sep=0pt, outer sep=0pt, scale=  1.14] at (113.78, 96.45) {0.04};

\node[text=drawColor,anchor=base,inner sep=0pt, outer sep=0pt, scale=  1.14] at (148.85, 96.45) {0.16};

\node[text=drawColor,anchor=base,inner sep=0pt, outer sep=0pt, scale=  1.14] at (183.92, 96.45) {0.78};

\node[text=drawColor,anchor=base,inner sep=0pt, outer sep=0pt, scale=  1.14] at (218.99, 96.45) {0.02};

\node[text=drawColor,anchor=base,inner sep=0pt, outer sep=0pt, scale=  1.14] at (113.78, 78.81) {0.02};

\node[text=drawColor,anchor=base,inner sep=0pt, outer sep=0pt, scale=  1.14] at (148.85, 78.81) {0.07};

\node[text=drawColor,anchor=base,inner sep=0pt, outer sep=0pt, scale=  1.14] at (183.92, 78.81) {0.71};

\node[text=drawColor,anchor=base,inner sep=0pt, outer sep=0pt, scale=  1.14] at (218.99, 78.81) {0.21};

\node[text=drawColor,anchor=base,inner sep=0pt, outer sep=0pt, scale=  1.14] at (113.78, 61.18) {0.00};

\node[text=drawColor,anchor=base,inner sep=0pt, outer sep=0pt, scale=  1.14] at (148.85, 61.18) {0.00};

\node[text=drawColor,anchor=base,inner sep=0pt, outer sep=0pt, scale=  1.14] at (183.92, 61.18) {0.03};

\node[text=drawColor,anchor=base,inner sep=0pt, outer sep=0pt, scale=  1.14] at (218.99, 61.18) {0.97};
\end{scope}
\begin{scope}
\path[clip] (245.54,150.34) rectangle (392.84,224.42);
\definecolor{drawColor}{RGB}{255,255,255}

\path[draw=drawColor,line width= 0.6pt,line join=round] (245.54,160.93) --
	(392.84,160.93);

\path[draw=drawColor,line width= 0.6pt,line join=round] (245.54,178.56) --
	(392.84,178.56);

\path[draw=drawColor,line width= 0.6pt,line join=round] (245.54,196.20) --
	(392.84,196.20);

\path[draw=drawColor,line width= 0.6pt,line join=round] (245.54,213.84) --
	(392.84,213.84);

\path[draw=drawColor,line width= 0.6pt,line join=round] (266.58,150.34) --
	(266.58,224.42);

\path[draw=drawColor,line width= 0.6pt,line join=round] (301.65,150.34) --
	(301.65,224.42);

\path[draw=drawColor,line width= 0.6pt,line join=round] (336.72,150.34) --
	(336.72,224.42);

\path[draw=drawColor,line width= 0.6pt,line join=round] (371.80,150.34) --
	(371.80,224.42);
\definecolor{fillColor}{RGB}{77,175,74}

\path[fill=fillColor,fill opacity=0.41] (249.04,205.02) rectangle (284.12,222.66);
\definecolor{fillColor}{RGB}{228,26,28}

\path[fill=fillColor,fill opacity=0.31] (284.12,205.02) rectangle (319.19,222.66);
\definecolor{fillColor}{RGB}{228,26,28}

\path[fill=fillColor,fill opacity=0.49] (319.19,205.02) rectangle (354.26,222.66);
\definecolor{fillColor}{RGB}{228,26,28}

\path[fill=fillColor,fill opacity=0.13] (354.26,205.02) rectangle (389.33,222.66);
\definecolor{fillColor}{RGB}{228,26,28}

\path[fill=fillColor,fill opacity=0.13] (249.04,187.38) rectangle (284.12,205.02);
\definecolor{fillColor}{RGB}{77,175,74}

\path[fill=fillColor,fill opacity=0.35] (284.12,187.38) rectangle (319.19,205.02);
\definecolor{fillColor}{RGB}{228,26,28}

\path[fill=fillColor,fill opacity=0.74] (319.19,187.38) rectangle (354.26,205.02);
\definecolor{fillColor}{RGB}{228,26,28}

\path[fill=fillColor,fill opacity=0.11] (354.26,187.38) rectangle (389.33,205.02);
\definecolor{fillColor}{RGB}{228,26,28}

\path[fill=fillColor,fill opacity=0.11] (249.04,169.74) rectangle (284.12,187.38);
\definecolor{fillColor}{RGB}{228,26,28}

\path[fill=fillColor,fill opacity=0.20] (284.12,169.74) rectangle (319.19,187.38);
\definecolor{fillColor}{RGB}{77,175,74}

\path[fill=fillColor,fill opacity=0.84] (319.19,169.74) rectangle (354.26,187.38);
\definecolor{fillColor}{RGB}{228,26,28}

\path[fill=fillColor,fill opacity=0.19] (354.26,169.74) rectangle (389.33,187.38);
\definecolor{fillColor}{RGB}{228,26,28}

\path[fill=fillColor,fill opacity=0.10] (249.04,152.11) rectangle (284.12,169.74);

\path[fill=fillColor,fill opacity=0.10] (284.12,152.11) rectangle (319.19,169.74);
\definecolor{fillColor}{RGB}{228,26,28}

\path[fill=fillColor,fill opacity=0.17] (319.19,152.11) rectangle (354.26,169.74);
\definecolor{fillColor}{RGB}{77,175,74}

\path[fill=fillColor,fill opacity=0.95] (354.26,152.11) rectangle (389.33,169.74);
\definecolor{drawColor}{RGB}{0,0,0}

\node[text=drawColor,anchor=base,inner sep=0pt, outer sep=0pt, scale=  1.14] at (266.58,209.92) {0.33};

\node[text=drawColor,anchor=base,inner sep=0pt, outer sep=0pt, scale=  1.14] at (301.65,209.92) {0.22};

\node[text=drawColor,anchor=base,inner sep=0pt, outer sep=0pt, scale=  1.14] at (336.72,209.92) {0.42};

\node[text=drawColor,anchor=base,inner sep=0pt, outer sep=0pt, scale=  1.14] at (371.80,209.92) {0.03};

\node[text=drawColor,anchor=base,inner sep=0pt, outer sep=0pt, scale=  1.14] at (266.58,192.28) {0.03};

\node[text=drawColor,anchor=base,inner sep=0pt, outer sep=0pt, scale=  1.14] at (301.65,192.28) {0.27};

\node[text=drawColor,anchor=base,inner sep=0pt, outer sep=0pt, scale=  1.14] at (336.72,192.28) {0.69};

\node[text=drawColor,anchor=base,inner sep=0pt, outer sep=0pt, scale=  1.14] at (371.80,192.28) {0.01};

\node[text=drawColor,anchor=base,inner sep=0pt, outer sep=0pt, scale=  1.14] at (266.58,174.64) {0.01};

\node[text=drawColor,anchor=base,inner sep=0pt, outer sep=0pt, scale=  1.14] at (301.65,174.64) {0.11};

\node[text=drawColor,anchor=base,inner sep=0pt, outer sep=0pt, scale=  1.14] at (336.72,174.64) {0.79};

\node[text=drawColor,anchor=base,inner sep=0pt, outer sep=0pt, scale=  1.14] at (371.80,174.64) {0.09};

\node[text=drawColor,anchor=base,inner sep=0pt, outer sep=0pt, scale=  1.14] at (266.58,157.01) {0.00};

\node[text=drawColor,anchor=base,inner sep=0pt, outer sep=0pt, scale=  1.14] at (301.65,157.01) {0.00};

\node[text=drawColor,anchor=base,inner sep=0pt, outer sep=0pt, scale=  1.14] at (336.72,157.01) {0.08};

\node[text=drawColor,anchor=base,inner sep=0pt, outer sep=0pt, scale=  1.14] at (371.80,157.01) {0.92};
\end{scope}
\begin{scope}
\path[clip] (245.54, 54.51) rectangle (392.84,128.59);
\definecolor{drawColor}{RGB}{255,255,255}

\path[draw=drawColor,line width= 0.6pt,line join=round] (245.54, 65.09) --
	(392.84, 65.09);

\path[draw=drawColor,line width= 0.6pt,line join=round] (245.54, 82.73) --
	(392.84, 82.73);

\path[draw=drawColor,line width= 0.6pt,line join=round] (245.54,100.37) --
	(392.84,100.37);

\path[draw=drawColor,line width= 0.6pt,line join=round] (245.54,118.01) --
	(392.84,118.01);

\path[draw=drawColor,line width= 0.6pt,line join=round] (266.58, 54.51) --
	(266.58,128.59);

\path[draw=drawColor,line width= 0.6pt,line join=round] (301.65, 54.51) --
	(301.65,128.59);

\path[draw=drawColor,line width= 0.6pt,line join=round] (336.72, 54.51) --
	(336.72,128.59);

\path[draw=drawColor,line width= 0.6pt,line join=round] (371.80, 54.51) --
	(371.80,128.59);
\definecolor{fillColor}{RGB}{77,175,74}

\path[fill=fillColor,fill opacity=0.45] (249.04,109.19) rectangle (284.12,126.83);
\definecolor{fillColor}{RGB}{228,26,28}

\path[fill=fillColor,fill opacity=0.23] (284.12,109.19) rectangle (319.19,126.83);
\definecolor{fillColor}{RGB}{228,26,28}

\path[fill=fillColor,fill opacity=0.54] (319.19,109.19) rectangle (354.26,126.83);
\definecolor{fillColor}{RGB}{228,26,28}

\path[fill=fillColor,fill opacity=0.12] (354.26,109.19) rectangle (389.33,126.83);
\definecolor{fillColor}{RGB}{228,26,28}

\path[fill=fillColor,fill opacity=0.15] (249.04, 91.55) rectangle (284.12,109.19);
\definecolor{fillColor}{RGB}{77,175,74}

\path[fill=fillColor,fill opacity=0.36] (284.12, 91.55) rectangle (319.19,109.19);
\definecolor{fillColor}{RGB}{228,26,28}

\path[fill=fillColor,fill opacity=0.71] (319.19, 91.55) rectangle (354.26,109.19);
\definecolor{fillColor}{RGB}{228,26,28}

\path[fill=fillColor,fill opacity=0.11] (354.26, 91.55) rectangle (389.33,109.19);
\definecolor{fillColor}{RGB}{228,26,28}

\path[fill=fillColor,fill opacity=0.12] (249.04, 73.91) rectangle (284.12, 91.55);
\definecolor{fillColor}{RGB}{228,26,28}

\path[fill=fillColor,fill opacity=0.16] (284.12, 73.91) rectangle (319.19, 91.55);
\definecolor{fillColor}{RGB}{77,175,74}

\path[fill=fillColor,fill opacity=0.80] (319.19, 73.91) rectangle (354.26, 91.55);
\definecolor{fillColor}{RGB}{228,26,28}

\path[fill=fillColor,fill opacity=0.24] (354.26, 73.91) rectangle (389.33, 91.55);
\definecolor{fillColor}{RGB}{228,26,28}

\path[fill=fillColor,fill opacity=0.10] (249.04, 56.28) rectangle (284.12, 73.91);

\path[fill=fillColor,fill opacity=0.10] (284.12, 56.28) rectangle (319.19, 73.91);
\definecolor{fillColor}{RGB}{228,26,28}

\path[fill=fillColor,fill opacity=0.14] (319.19, 56.28) rectangle (354.26, 73.91);
\definecolor{fillColor}{RGB}{77,175,74}

\path[fill=fillColor,fill opacity=0.98] (354.26, 56.28) rectangle (389.33, 73.91);
\definecolor{drawColor}{RGB}{0,0,0}

\node[text=drawColor,anchor=base,inner sep=0pt, outer sep=0pt, scale=  1.14] at (266.58,114.09) {0.37};

\node[text=drawColor,anchor=base,inner sep=0pt, outer sep=0pt, scale=  1.14] at (301.65,114.09) {0.14};

\node[text=drawColor,anchor=base,inner sep=0pt, outer sep=0pt, scale=  1.14] at (336.72,114.09) {0.47};

\node[text=drawColor,anchor=base,inner sep=0pt, outer sep=0pt, scale=  1.14] at (371.80,114.09) {0.02};

\node[text=drawColor,anchor=base,inner sep=0pt, outer sep=0pt, scale=  1.14] at (266.58, 96.45) {0.06};

\node[text=drawColor,anchor=base,inner sep=0pt, outer sep=0pt, scale=  1.14] at (301.65, 96.45) {0.28};

\node[text=drawColor,anchor=base,inner sep=0pt, outer sep=0pt, scale=  1.14] at (336.72, 96.45) {0.66};

\node[text=drawColor,anchor=base,inner sep=0pt, outer sep=0pt, scale=  1.14] at (371.80, 96.45) {0.01};

\node[text=drawColor,anchor=base,inner sep=0pt, outer sep=0pt, scale=  1.14] at (266.58, 78.81) {0.02};

\node[text=drawColor,anchor=base,inner sep=0pt, outer sep=0pt, scale=  1.14] at (301.65, 78.81) {0.07};

\node[text=drawColor,anchor=base,inner sep=0pt, outer sep=0pt, scale=  1.14] at (336.72, 78.81) {0.75};

\node[text=drawColor,anchor=base,inner sep=0pt, outer sep=0pt, scale=  1.14] at (371.80, 78.81) {0.15};

\node[text=drawColor,anchor=base,inner sep=0pt, outer sep=0pt, scale=  1.14] at (266.58, 61.18) {0.00};

\node[text=drawColor,anchor=base,inner sep=0pt, outer sep=0pt, scale=  1.14] at (301.65, 61.18) {0.00};

\node[text=drawColor,anchor=base,inner sep=0pt, outer sep=0pt, scale=  1.14] at (336.72, 61.18) {0.04};

\node[text=drawColor,anchor=base,inner sep=0pt, outer sep=0pt, scale=  1.14] at (371.80, 61.18) {0.95};
\end{scope}
\begin{scope}
\path[clip] ( 92.74,128.59) rectangle (240.04,144.84);
\definecolor{fillColor}{gray}{0.85}

\path[fill=fillColor] ( 92.74,128.59) rectangle (240.04,144.84);
\definecolor{drawColor}{gray}{0.10}

\node[text=drawColor,anchor=base,inner sep=0pt, outer sep=0pt, scale=  0.80] at (166.39,133.96) {Dense-Low-Level};
\end{scope}
\begin{scope}
\path[clip] (245.54,128.59) rectangle (392.84,144.84);
\definecolor{fillColor}{gray}{0.85}

\path[fill=fillColor] (245.54,128.59) rectangle (392.84,144.84);
\definecolor{drawColor}{gray}{0.10}

\node[text=drawColor,anchor=base,inner sep=0pt, outer sep=0pt, scale=  0.80] at (319.19,133.96) {Lstm-Rule};
\end{scope}
\begin{scope}
\path[clip] ( 92.74,224.42) rectangle (240.04,240.67);
\definecolor{fillColor}{gray}{0.85}

\path[fill=fillColor] ( 92.74,224.42) rectangle (240.04,240.67);
\definecolor{drawColor}{gray}{0.10}

\node[text=drawColor,anchor=base,inner sep=0pt, outer sep=0pt, scale=  0.80] at (166.39,229.79) {Rule};
\end{scope}
\begin{scope}
\path[clip] (245.54,224.42) rectangle (392.84,240.67);
\definecolor{fillColor}{gray}{0.85}

\path[fill=fillColor] (245.54,224.42) rectangle (392.84,240.67);
\definecolor{drawColor}{gray}{0.10}

\node[text=drawColor,anchor=base,inner sep=0pt, outer sep=0pt, scale=  0.80] at (319.19,229.79) {Bayes};
\end{scope}
\begin{scope}
\path[clip] (  0.00,  0.00) rectangle (398.34,246.17);
\definecolor{drawColor}{gray}{0.20}

\path[draw=drawColor,line width= 0.6pt,line join=round] (113.78, 51.76) --
	(113.78, 54.51);

\path[draw=drawColor,line width= 0.6pt,line join=round] (148.85, 51.76) --
	(148.85, 54.51);

\path[draw=drawColor,line width= 0.6pt,line join=round] (183.92, 51.76) --
	(183.92, 54.51);

\path[draw=drawColor,line width= 0.6pt,line join=round] (218.99, 51.76) --
	(218.99, 54.51);
\end{scope}
\begin{scope}
\path[clip] (  0.00,  0.00) rectangle (398.34,246.17);
\definecolor{drawColor}{RGB}{0,0,0}

\node[text=drawColor,rotate= 20.00,anchor=base east,inner sep=0pt, outer sep=0pt, scale=  1.00] at (116.13, 43.09) {Speaker};

\node[text=drawColor,rotate= 20.00,anchor=base east,inner sep=0pt, outer sep=0pt, scale=  1.00] at (151.21, 43.09) {Addressee};

\node[text=drawColor,rotate= 20.00,anchor=base east,inner sep=0pt, outer sep=0pt, scale=  1.00] at (186.28, 43.09) {Side-Participant};

\node[text=drawColor,rotate= 20.00,anchor=base east,inner sep=0pt, outer sep=0pt, scale=  1.00] at (221.35, 43.09) {Non-Participant};
\end{scope}
\begin{scope}
\path[clip] (  0.00,  0.00) rectangle (398.34,246.17);
\definecolor{drawColor}{gray}{0.20}

\path[draw=drawColor,line width= 0.6pt,line join=round] (266.58, 51.76) --
	(266.58, 54.51);

\path[draw=drawColor,line width= 0.6pt,line join=round] (301.65, 51.76) --
	(301.65, 54.51);

\path[draw=drawColor,line width= 0.6pt,line join=round] (336.72, 51.76) --
	(336.72, 54.51);

\path[draw=drawColor,line width= 0.6pt,line join=round] (371.80, 51.76) --
	(371.80, 54.51);
\end{scope}
\begin{scope}
\path[clip] (  0.00,  0.00) rectangle (398.34,246.17);
\definecolor{drawColor}{RGB}{0,0,0}

\node[text=drawColor,rotate= 20.00,anchor=base east,inner sep=0pt, outer sep=0pt, scale=  1.00] at (268.94, 43.09) {Speaker};

\node[text=drawColor,rotate= 20.00,anchor=base east,inner sep=0pt, outer sep=0pt, scale=  1.00] at (304.01, 43.09) {Addressee};

\node[text=drawColor,rotate= 20.00,anchor=base east,inner sep=0pt, outer sep=0pt, scale=  1.00] at (339.08, 43.09) {Side-Participant};

\node[text=drawColor,rotate= 20.00,anchor=base east,inner sep=0pt, outer sep=0pt, scale=  1.00] at (374.15, 43.09) {Non-Participant};
\end{scope}
\begin{scope}
\path[clip] (  0.00,  0.00) rectangle (398.34,246.17);
\definecolor{drawColor}{RGB}{0,0,0}

\node[text=drawColor,anchor=base east,inner sep=0pt, outer sep=0pt, scale=  1.00] at ( 87.79,157.48) {Non-Participant};

\node[text=drawColor,anchor=base east,inner sep=0pt, outer sep=0pt, scale=  1.00] at ( 87.79,175.12) {Side-Participant};

\node[text=drawColor,anchor=base east,inner sep=0pt, outer sep=0pt, scale=  1.00] at ( 87.79,192.76) {Addressee};

\node[text=drawColor,anchor=base east,inner sep=0pt, outer sep=0pt, scale=  1.00] at ( 87.79,210.39) {Speaker};
\end{scope}
\begin{scope}
\path[clip] (  0.00,  0.00) rectangle (398.34,246.17);
\definecolor{drawColor}{gray}{0.20}

\path[draw=drawColor,line width= 0.6pt,line join=round] ( 89.99,160.93) --
	( 92.74,160.93);

\path[draw=drawColor,line width= 0.6pt,line join=round] ( 89.99,178.56) --
	( 92.74,178.56);

\path[draw=drawColor,line width= 0.6pt,line join=round] ( 89.99,196.20) --
	( 92.74,196.20);

\path[draw=drawColor,line width= 0.6pt,line join=round] ( 89.99,213.84) --
	( 92.74,213.84);
\end{scope}
\begin{scope}
\path[clip] (  0.00,  0.00) rectangle (398.34,246.17);
\definecolor{drawColor}{RGB}{0,0,0}

\node[text=drawColor,anchor=base east,inner sep=0pt, outer sep=0pt, scale=  1.00] at ( 87.79, 61.65) {Non-Participant};

\node[text=drawColor,anchor=base east,inner sep=0pt, outer sep=0pt, scale=  1.00] at ( 87.79, 79.29) {Side-Participant};

\node[text=drawColor,anchor=base east,inner sep=0pt, outer sep=0pt, scale=  1.00] at ( 87.79, 96.93) {Addressee};

\node[text=drawColor,anchor=base east,inner sep=0pt, outer sep=0pt, scale=  1.00] at ( 87.79,114.56) {Speaker};
\end{scope}
\begin{scope}
\path[clip] (  0.00,  0.00) rectangle (398.34,246.17);
\definecolor{drawColor}{gray}{0.20}

\path[draw=drawColor,line width= 0.6pt,line join=round] ( 89.99, 65.09) --
	( 92.74, 65.09);

\path[draw=drawColor,line width= 0.6pt,line join=round] ( 89.99, 82.73) --
	( 92.74, 82.73);

\path[draw=drawColor,line width= 0.6pt,line join=round] ( 89.99,100.37) --
	( 92.74,100.37);

\path[draw=drawColor,line width= 0.6pt,line join=round] ( 89.99,118.01) --
	( 92.74,118.01);
\end{scope}
\begin{scope}
\path[clip] (  0.00,  0.00) rectangle (398.34,246.17);
\definecolor{drawColor}{RGB}{0,0,0}

\node[text=drawColor,anchor=base,inner sep=0pt, outer sep=0pt, scale=  1.00] at (242.79,  7.44) {Predicted};
\end{scope}
\begin{scope}
\path[clip] (  0.00,  0.00) rectangle (398.34,246.17);
\definecolor{drawColor}{RGB}{0,0,0}

\node[text=drawColor,rotate= 90.00,anchor=base,inner sep=0pt, outer sep=0pt, scale=  1.00] at ( 12.39,139.47) {Role};
\end{scope}
\end{tikzpicture}

            }
          };}
        \end{tikzpicture}
      }
    \end{column}%
    \hspace{-10pt}
    \begin{column}{.45\textwidth}
      \footnotesize
      \vspace{10pt}
      \begin{itemize}[label=-]
        \item<1-> Best \(Dense\) uses low-level features
        \item<2-> Best \(Lstm\) uses high-level features
        \item<3->[] Confusion Matrices
        \item<4-> Good prediction of \(Non\text{-}Participants\)
        \item<5-> Bias of \(Rule\) for \(Addressee\)
        \item<6-> Bias of Others for \(Side\text{-}Participants\)
        \item<7-> \(Dense\) optimizes for \(Non\text{-}Participants\)
        \item<8-> \(Lstm\) better by a small margin
      \end{itemize}
      \vspace{10pt}
      %\begin{itemize}[label=-]
      %  \item<9->[\textcolor{mygreen}{\faCheckCircle}] Simple Models
      %  \item<10->[\textcolor{myred}{\faTimesCircle}] Low-Level Features
      %  \item<11->[\textcolor{mygreen}{\faCheckCircle}] Time Sequences
      %\end{itemize}
    \end{column}%
  \end{columns}
  \pnote{14-5 \\ this is recall!!}
\end{frame}
\begin{frame}{Summary RQ4}
  \begin{itemize}
    \item[\textcolor{mygreen}{\faCheckCircle}]<1-> \textcolor{myblue}{Simple Models} can recognize conversational roles on high-level features.
    \item[\textcolor{myred}{\faTimesCircle}]<2-> \textcolor{myblue}{Low-Level Features} do not further enhance the recognition.
    \item[\textcolor{mygreen}{\faCheckCircle}]<3-> \textcolor{myblue}{Time Sequences} can be observed to further enhance the recognition.
  \end{itemize}
  \pnote{14-5}
\end{frame}
