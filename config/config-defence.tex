% required for lualatex in 2018 due to microtype bug
\usepackage{luatexbase}

% show overfull boxes
\overfullrule=2cm

\PassOptionsToPackage{british}{babel}
\usepackage{babel}

\PassOptionsToPackage{%
    backend=biber,%
    language=auto,%
    style=alphabetic,%
    maxnames=6,%
    minnames=6,%
    backref=true,%
    urldate=iso,%
    seconds=true,%
    date=iso,%
    datamodel=custom,%
    datecirca=true,%
    dateuncertain=true,%
}{biblatex}
\usepackage{biblatex}
\setcounter{biburllcpenalty}{2000}
\setcounter{biburlucpenalty}{2000}
\DeclareBibliographyAlias{software}{online}
\DeclareLanguageMapping{english}{american-apa}
\DeclareSourcemap{
    \maps[datatype=bibtex]{
        \map[overwrite=true]{
            \step[fieldsource=author, match=Richter, final]
            \step[fieldset=keywords, fieldvalue=ownpub]
        }
        \map{
            \step[fieldsource=organization, final]
            \step[fieldset=usera, origfieldval]
        }
        \map{
            \pertype{standard}
            \step[typesource=standard, typetarget=manual]
            \step[fieldset=keywords, fieldvalue={,type:standard}, append]
        }
        % remove fields that are always useless
        \map{
            \step[fieldset=abstract, null]
            \step[fieldset=pagetotal, null]
        }
        % remove URLs for everything apart from online and software sources
        \map{
            \pernottype{software}
            \pernottype{online}
            \pernottype{report}
            \pernottype{techreport}
            \pernottype{standard}
            \pernottype{manual}
            \pernottype{misc}
            \step[fieldset=url, null]
            \step[fieldset=urldate, null]
        }
        \map{
            \pertype{inproceedings}
            % remove redundant conference information
            \step[fieldset=venue, null]
            \step[fieldset=eventdate, null]
            \step[fieldset=eventtitle, null]
            % do not show ISBN for proceedings
            \step[fieldset=isbn, null]
            % Citavi bug
            \step[fieldset=volume, null]
        }
        \map{
            \pertype{software}
            %\step[fieldset=author, null]
        }
    }
}
\newrobustcmd*{\citefullauthor}{\AtNextCite{\DeclareNameAlias{labelname}{given-family}}\citeauthor}
\newrobustcmd*{\Citefullauthor}{\AtNextCite{\DeclareNameAlias{labelname}{given-family}}\Citeauthor}
% use organization as a replacement for author or editor
\DeclareLabelalphaTemplate{
    \labelelement{
        \field[final]{shorthand}
        \field{label}
        \field[strwidth=3,strside=left,ifnames=1]{labelname}
        \field[strwidth=1,strside=left]{labelname}
        \field[strwidth=3]{usera}
    }
    \labelelement{
        \field[strwidth=2,strside=right]{year}
    }
}
\newcommand{\citesoftware}[1]{\citetitle{#1} by \citefullauthor*{#1} \autocite{#1}}
\newcommand{\citesoft}[1]{\citetitle{#1} \autocite{#1}}
\newcommand{\citewithauthor}[1]{\citeauthor*{#1}~\cite{#1}}

% configure back references
\DefineBibliographyStrings{english}{%
  backrefpage = {used on: p\adddot},
  backrefpages = {used on: pp\adddot},
}
\renewbibmacro*{finentry}{\iflistundef{pageref}{}{\renewcommand{\finentrypunct}{}}\finentry}
\DeclareFieldFormat{pagerefformat}{\mkbibemph{#1}}
\renewbibmacro*{pageref}{%
  \iflistundef{pageref}
    {\adddot}
    {\setunit{\adddot\addspace}\printtext[pagerefformat]{%
       \ifnumgreater{\value{pageref}}{1}
         {\bibstring{backrefpages}\ppspace}
         {\bibstring{backrefpage}\ppspace}%
       \printlist[pageref][-\value{listtotal}]{pageref}}}}
\renewcommand*{\bibfont}{\small}
\DeclareBibliographyDriver{software}{%
  \noindent\usebibmacro{author}%
  \newunit
  \usebibmacro{title}%
  \newunit
  \usebibmacro{url} \usebibmacro{urldate}
  \iffieldundef{addendum}{}{\newunit \printfield{addendum}}%
  \newunit
  \usebibmacro{pageref}
  \usebibmacro{finentry}
}

\bibliography{bib/library}
\bibliography{bib/software}

% try to prevent more hboxes in the bibliography
\usepackage{etoolbox}
\apptocmd{\sloppy}{\hbadness 100000\relax}{}{} % chktex 1

\AtBeginEnvironment{tabular}{\footnotesize}

% must be loaded after hyperref
\usepackage[style=indexgroup,nolist,xindy,acronyms]{glossaries-extra}
\usepackage{glossaries-prefix}
\RestoreAcronyms

% additional forms of glossary entries
% attributive form
\glsaddkey
 {attributive}% key
 {}% default value
 {\glsentryattributive}% no link cs
 {\Glsentryattributive}% no link ucfirst cs
 {\glsatt}% link cs
 {\Glsatt}% link ucfirst cs
 {\GLSatt}% link all caps cs
% some alternative form
\glsaddkey
 {alternative}% key
 {}% default value
 {\glsentryalt}% no link cs
 {\Glsentryalt}% no link ucfirst cs
 {\glsalt}% link cs
 {\Glsalt}% link ucfirst cs
 {\GLSalt}% link all caps cs


\GlsXtrEnableLinkCounting[section]{general,acronym}
% disable hyperlink if link count is greater than 1:
\renewcommand*{\glslinkpresetkeys}{%
 \ifnum\GlsXtrLinkCounterValue{\glslabel}>1
  \setkeys{glslink}{hyper=false}%
 \fi
}

\renewcommand{\glsxtrpostdescacronym}{.}
\newcommand{\glsxtrpostdescdualacronym}{.}
% start linking to entries now that they are defined
\newcommand{\glsxtrpostlinkdualacronym}{\glsadd{main-\glslabel}}
\renewcommand{\glstreeitem}{%
  \parindent0pt\par\hangindent40pt
  \everypar{\parindent50pt\hangindent40pt}}
\renewcommand{\glstreepredesc}{\par\nopagebreak
  \glstreeitem\parindent\hangindent}
\renewcommand{\glsnamefont}[1]{\textnormal{\itshape #1}} % chktex 1 chktex 6
\renewcommand*{\glsgroupheading}[1]{\parindent0pt\textcolor{Maroon}{\textnormal{\glsgetgrouptitle{#1}}}%
  \par\nopagebreak
}
\GlsXtrEnablePreLocationTag{\textit{used on: p.~}}{\textit{used on: pp.~}}
\renewcommand{\GlsXtrFormatLocationList}{\textit}
\newcommand*{\newdualentrypl}[6][]{%
  \newglossaryentry{main-#2}{%
  name={#4 (\gls{#2})},%
  plural={#5},%
  text={#3\glsadd{main-#2}},%
  description={{#6}},%
  #1
  }%

  \newglossaryentry{#2}{
  type=\acronymtype,
  category=dualacronym,
  first={#4 (#3)},
  long={#4},
  short={#3},
  shortplural={#3s},
  longplural={#4s},
  name={#3},
  description={\glslink{main-#2}{\capitalisewords{#4}}},
  #1
  }%
}
\newcommand*{\newdualentry}[5][]{\newdualentrypl{#2}{#3}{#4}{#4s}{#5}}
\newcommand*{\newaltglossentry}[4][]{%
  \newglossaryentry{#2}{name={#3},%
  text={#3\glsadd{#2}},%
  description={{Also known as \gls{#4}}},%
  #1
  }%
}


\loadglsentries{frontback/Glossary}
\makeglossaries{}

\usepackage[useregional]{datetime2}
\PassOptionsToPackage{activate={true,nocompatibility},final,tracking=true}{microtype}
\usepackage{fnpct}
\usepackage{soul}

% command for hyphenated type writer font
\makeatletter
\DeclareRobustCommand\ttfamily
        {\not@math@alphabet\ttfamily\mathtt
         \fontfamily\ttdefault\selectfont\hyphenchar\font=-1\relax}
\makeatother
\DeclareTextFontCommand{\texthtt}{\ttfamily\hyphenchar\font=45\relax}

\hypersetup{%
    %draft, % = no hyperlinking at all (useful in b/w printouts)
    colorlinks=true, linktocpage=true, pdfstartpage=3, pdfstartview=FitV,%
    % uncomment the following line if you want to have black links (e.g., for printing)
    colorlinks=false, linktocpage=false, pdfstartpage=3, pdfstartview=FitV, pdfborder={0 0 0},
    urlcolor=mybrown, linkcolor=myblue, citecolor=mygreen,%
    %urlcolor=Black, linkcolor=Black, citecolor=Black,%
    breaklinks=true, pdfpagemode=UseNone, pageanchor=true, pdfpagemode=UseOutlines,%
    plainpages=false, bookmarksnumbered, bookmarksopen=true, bookmarksopenlevel=1,%
    hypertexnames=true, pdfhighlight=/O,%nesting=true,%frenchlinks,%
    pdftitle={\myTitle},%
    pdfauthor={\myName},%
    pdfsubject={PhD Thesis},%
    pdfkeywords={},%
    pdfcreator={LuaTeX},%
}

\usepackage{scrhack}
\usepackage[binary-units=true]{siunitx}
\sisetup{detect-all}
\usepackage{setspace}
\usepackage{calc}
\usepackage{array}
\usepackage{fontawesome}

\usepackage[nameinlink,noabbrev,capitalise]{cleveref}
\crefformat{part}{#2Part \MakeUppercase{#1}#3}
\crefname{hyp1}{Goal}{Goals}
\crefname{hyp2}{RQ}{RQs}
\crefrangelabelformat{hyp2}{#3#1#4--#5#2#6}
\Crefname{hyp3}{Claim}{Claims}

\usepackage{ragged2e}
\usepackage{placeins}
\usepackage{subcaption}
\usepackage{fancyvrb}
\usepackage{interval}
\usepackage[inline]{enumitem}

% prevent a page break before a list
\makeatletter
\newcommand\mynobreakpar{\par\nobreak\@afterheading}
\makeatother

% allow splitting math across pages
\allowdisplaybreaks

\usepackage{tikz}
\usepackage{tabulary}
\usepackage{eurosym}
\usepackage{pgfgantt}
\usepackage{ellipsis}
\usepackage{datatool}

\setcounter{secnumdepth}{2}
\setcounter{tocdepth}{2}
\graphicspath{{figures/},{generated/},{data/}}

\makeatletter
\let\thetitle\@title
\let\thesubtitle\@subtitle
\let\theauthor\@author
\makeatother

% Signature and date
\newcommand*{\SignatureAndDate}[2][3cm]{%
	\vspace*{#1}
    \par\noindent\makebox[.45\textwidth]{\hrulefill} \hfill\makebox[.45\textwidth]{\hrulefill}%
    \par\noindent\makebox[.45\textwidth][l]{#2}      \hfill\makebox[.45\textwidth][l]{Place, Date}%
}%

% special writing and not decided
\newcommand{\naive}{na{\"i}ve}
\newcommand{\Naive}{Na{\"i}ve}
\newcommand{\visavis}{vis-{\`a}-vis}
\newcommand{\yele}{Y{\`e}l{\^i} Dnye}
\newcommand{\cramv}{Cram\'er's \~V}
\newcommand{\chisq}{Pearson's chi-square}
\newcommand{\fish}{Fisher's exact}

% formatting
\def\code#1{\textcolor{code}{\texttt{#1}}}

% create a special environment for colored tables. allows non-colored tabled and
% prevents colored text in pdf_tex exported from inkscape in bionic
\newenvironment{colored_table}[1]
{
    \rowcolors{1}{color_table_one}{color_table_two}
    \begin{table}[#1]
}
{
    \end{table}
    \rowcolors{1}{white}{white}
}

% definition marks on margin
\newcommand*{\mnote}[2]{
\marginnote
[ % left pages
  {\setlength{\tabcolsep}{2pt}
  \begin{tabular}{>{\RaggedLeft\hspace{0pt}}p{.769\marginparwidth}>{\RaggedLeft\hspace{-2pt}}p{1em}}
    #1 & \textcolor{halfgray}{#2}\\
  \end{tabular}
  }
]
{ % right pages
  {\setlength{\tabcolsep}{2pt}
  \begin{tabular}{>{\RaggedRight\hspace{-2pt}}p{1em}>{\RaggedRight\hspace{0pt}}p{.769\marginparwidth}}
    \textcolor{halfgray}{#2} & #1\\
  \end{tabular}
  }
}
% voffset
[-\baselineskip]
}
\newcommand{\newdefmanual}[2]{\mnote{\begin{footnotesize}#1\end{footnotesize}}{\faEdit}\emph{#2}}
\newcommand{\newdefgls}[1]{\newdefmanual{\gls+{#1}\glsreset{#1}}{\gls+{#1}}}
\newcommand{\newdefGls}[1]{\newdefmanual{\gls+{#1}\glsreset{#1}}{\Gls+{#1}}}
\newcommand{\newdefglspl}[1]{\newdefmanual{\gls+{#1}\glsreset{#1}}{\glspl+{#1}}}
\newcommand{\newdefGlspl}[1]{\newdefmanual{\gls+{#1}\glsreset{#1}}{\Glspl+{#1}}}
%\newcommand{\rqnote}[2]{\marginnote{\begin{footnotesize}\hyperref[#1]{\textcolor{halfgray}{\Cref*{#1}:}} #2\end{footnotesize}}}
%\newcommand{\rqnote}[2]{\marginnote{\justify\begin{footnotesize}\hyperref[#1]{\textcolor{halfgray}{\Cref*{#1}:}} #2\end{footnotesize}}[-1.65\baselineskip]}
%\newcommand{\rqnote}[2]{
%  \begin{wrapfigure}{r}[\marginparwidth+\marginparsep]{2.\marginparwidth+\marginparsep}
%          \hyperref[#1]{\textcolor{halfgray}{\Cref*{#1}:}} #2
%  \end{wrapfigure}
%}
\usepackage{manyfoot}
\DeclareNewFootnote{A}
\newcommand{\rqnote}[2]{\FootnoteA{\textcolor{halfgray}{\tiny\faRefresh}}{\hyperref[#1]{\textcolor{halfgray}{\Cref*{#1}:}} #2}}

% load and use data from an ini file
\DTLsetseparator{ = }
\newcommand{\iniload}[2]{\DTLloaddb[noheader, keys={key,value}]{#1}{#2}}
\newcommand{\inidata}[2]{\DTLfetch{#1}{key}{#2}{value}}
\newcommand{\inisi}[3]{\(\DTLfetch{#1}{key}{#2}{value}\,\si{#3}\)}

\hyphenation{%
    pre-sen-ter
    vi-de-os
    con-ver-sa-tion-al
    au-ton-o-mous
    au-ton-o-mously
    ro-bot-ics
    ro-bot
    ro-bots
    mod-el
    over-view
}

\newcommand{\bcite}[1]{\footnote[frame]{\tiny{\citeauthor{#1}: \citetitle{#1}}}}
\newcommand{\obcite}[2]{\footnote[frame]{\onslide<#1>{\tiny{\citeauthor{#2}: \citetitle{#2}}}}}

\usepackage{luatex85}
\usepackage{appendixnumberbeamer}
\usepackage{pgfpages}
\usepackage{pdfpages}
\usepackage{comment}
%\usepackage{multimedia}
\usepackage{pdfpc-movie}
\usepackage[duration=45]{config/pdfpcnotes}
\colorlet{bgcolor}{bgcolorframe}
